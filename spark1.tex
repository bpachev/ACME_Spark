\documentclass[nociteref]{SIAM-GH-book}

\usepackage{hyperref}
\usepackage{import}
\usepackage{amsmath, amsfonts, amscd, amssymb}
\usepackage{epsfig}
\usepackage{graphicx}
\usepackage{url}
\usepackage{mathrsfs}
\usepackage{makeidx}
\usepackage{multicol}
\usepackage{algorithmicx}
\usepackage[plain]{algorithm}
\usepackage[noend]{algpseudocode}
\usepackage{color}
\usepackage{verbatim}
\usepackage{listings}
\usepackage{float}
\usepackage{paralist}
\usepackage{caption}
\usepackage{subcaption}
\usepackage{bbm}
\usepackage{textcomp}
\usepackage{tikz}
\usepackage[framemethod=tikz]{mdframed}
\usepackage[style=alphabetic,refsection=chapter,backref=true]{biblatex}
\usepackage{mathtools}
\usepackage{xcolor}

\input{command}

\usetikzlibrary{automata,positioning, arrows, backgrounds, calendar, chains, decorations,
	matrix, mindmap, patterns, petri, shadows, shapes.geometric,
	trees}

\begin{document}



\lab{Introduction to Spark}{Introduction to Spark}
\label{lab:rivercrossing}
\objective{This lab gives an introduction to parallelizing algorithms in a cluster environment using Apache Spark} 
\section{Intro to Cluster Computing}
   - What and why
\subsection*{Hadoop}
   - What and why, as well as limitations
\subsection*{Spark}
   - How it overcomes Hadoop's limitations
(Note: I-III are just writing, no problems, which from here will be marked with numbers like (1))
\section{Mapping/ Map-Reduce}
   - Short Description
   (1) Creating RDD's (Resilient Distributed Data Sets)

   (2) RDD Actions (Problem like count, sum, and max computations)
   (input table from link above on actions)

   (3) RDD Transformations (Filter, map)
\begin{problem}
   (4) Basic Map-Reduce Problem
\end{problem}
\begin{problem}
   (5) Interesting/More Difficult Map-Reduce Problem
\end{problem}

This is the part after all the problems in the lab, which no one really reads.

\end{document}
